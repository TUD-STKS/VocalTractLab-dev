\documentclass[]{article}
\usepackage{bm}
\usepackage{amsmath}
\usepackage{amsfonts} 
\usepackage{graphicx}
\usepackage[utf8]{inputenc}
\usepackage[T1]{fontenc}
\usepackage{mathtools}
%\usepackage{breqn}
\usepackage[left=2cm, right=2cm, top=2cm, bottom=2cm]{geometry}
\usepackage{soul}
\usepackage{color}
\usepackage{hyperref}
\usepackage{bold-extra}
\usepackage{tcolorbox}

\newcommand*{\transp}[2][-3mu]{\ensuremath{\mskip1mu\prescript{\smash{\mathrm t\mkern#1}}{}{\mathstrut#2}}}
\newcommand{\mathcolorbox}[2]{\colorbox{#1}{$\displaystyle #2$}}
\newcommand\tab[1][1cm]{\hspace*{#1}}

%opening
\title{User guide for VocalTractLab3D}
\author{R{\'e}mi Blandin}

\begin{document}
	\maketitle
	
	\section{Introduction}
	
	VocalTractLab3D is a special version of the articulatory synthesizer
	VocalTractLab 2.3 which integrates a module that performs 3D acoustic simulations. The other modules are, to some very little differences the same as the original VocalTractLab 2.3 and the 
	reader is referred to the manual of VocalTractLab 2.3 to learn
	how to use them.
	The 3D acoustic simulations are performed with a frequency domain
	multimodal method which have been designed to be particularly 
	fast and accurate. The details of this simulation method are 
	provided in an article \hl{[CITE ACCESS HERE]}.
	
	\subsection{What are 3D acoustic simulations?}
	
	\textbf{What are vocal tract acoustic simulations?}\\ 
	Vocal tract acoustic simulations consists in describing how acoustic waves travel
	inside the vocal tract volume, are reflected at the discontinuities, such as 
	changes of cross-section or the mouth opening and create resonances. 
	This allows one to compute transfer functions, as an exemple between the acoustic volume flow created at the vocal folds and the acoustic pressure radiated in front of the lips. It can also be used to 
	compute the acoustic field, which describes the variations of the  
	properties of the acoustic waves (acoustic pressure and particle velocity) over space.
	
	\textbf{What is specific to vocal tract acoustics?}\\
	The vocal tract has an elongated shape in which the acoustic waves 
	are guided to travel mainly along its length. One can say that the
	waves are guided by the vocal tract walls and thus, from the point
	of view of wave propagation, the vocal tract can be called a 
	\emph{waveguide} \hl{check if wave-guide or waveguide}.
	This property of the vocal tract makes it easy to approximate the 
	propagation of acoustic waves as a single value of acoustic 
	pressure varying along its length, thus neglecting transverse 
	variations of the acoustic field. This have lead to 1D simulation 
	methods and electrical analogies which are very widely used to 
	simulate vocal tract acoustics \hl{[PUT RELEVANT REFERENCES HERE]}.
	
	\textbf{What is 3D vocal tract acoustics?}\\
	Even though not very important below 4-5~kHz, 
	the acoustic field has transverse variations, and thus, varies in
	all the three dimensions of space. At low frequency these variations
	appear as a curvature of the acoustic field related to variations
	of cross-sectional dimensions \hl{Illustrate this with acoustic field pictures}. At higher frequency, the 3D nature of the acoustic field is more obvious as transverse resonances can be observed
	\hl{illustrate that}.
	
	\textbf{What does it changes to account for the 3D acoustic in comparison with using a 1D simplifying assumption?}\\
	The impact of accounting for the 3D nature of the acoustic field 
	inside the vocal tract is rather limited up to about 3~kHz. 
	It consists mainly in small changes in the resonance properties
	(frequency, amplitude and bandwidth). From 3~kHz on, the changes 
	in the resonance properties can be more substantial and above 
	4-5~kHz the transverse resonances can induce zeros and additional 
	peaks in the transfer function. \hl{Illustrate with a TF 3D vs 1D}
	
	\subsection{What can VocalTractLab3D do and not do?}
	
	\textbf{What VocalTractLab3D can do?}
	\begin{itemize}
		\item Compute the transfer function between the volume velocity
		at the location of the vocal folds and one are several points inside or outside the vocal tract.
		\item Compute the transfer function between the acoustic pressure on a transverse plane anywhere inside the vocal tract and one or several points inside or outside the vocal tract.
		This is useful to emulate noise generation by aeroacoustic 
		sound sources.
		\item Compute the input impedance of the plane mode at the 
		location of the vocal folds.
		\item Compute and visualize the transverse modes.
		\item Compute the acoustic field at a specific frequency in the sagittal plane and in transverse planes anywhere inside the vocal tract.
		\item Compute the above listed quantities for vocal tract 
		geometries generated with the articulatory model implemented 
		in VocalTractLab3D or vocal tract geometries imported from 
		external .csv file coded in a specific format. Note that in 
		this last case any waveguide geometry can be imported and not only vocal tract geometry (e.g. airways of other animals, wind instruments ...). However, the parameters of VocalTractLab3D 
		are optimized for human vocal tracts and may not be optimal for
		other applications.
		\item Synthesize vowel and fricative sounds.
	\end{itemize}

	\textbf{What VocalTractLab3D cannot do?} \\
	The simulation method implemented in VocalTractLab3D has been 
	designed primarily to be efficient, and this comes with some 
	limitations regarding the geometries which can be simulated.
	However, it is to be noted that some of the geometrical simplifications are also used with simulation methods, such as finite
	elemnents, in order to reduce the computation time, or simplify the task of describing the geometry.
	VocalTractLab3D cannot:
	\begin{itemize}
		\item Simulate continuous cross-sectional shape variations within the segments of the vocal tract geometry. The cross-sectional shape can be scaled to account for area variation, but the shape reamins constant.
		\item Simulate lip shapes. The mouth opening is contained 
		inside a plane, and thus, the 3D lip shape cannot be simulated.
		\item Simulate branches such as piriform sinus or the nasal 
		cavity.
		\item Similate the diffraction by the head and the torso outside
		of the vocal tract.
		\item Perform time domain simulation.
	\end{itemize}
	Some of the limitations listed may be overcomed with future developments of the simulation method used (3D lip shape, branches 
	and diffraction by the head and torso), and some are inherent to 
	the simulation method used (constant cross-sectional shape in the 
	segments, and frequency domain simulations).
	
	\subsection{Installation requirements}
	
	\subsection{How to cite VocalTractLab3D?}
	
	\section{Interface}
	
	\hl{SHOW THE INTERFACE}
	
	The 3D acoustic simulation can be performed using the 
	
	

\end{document}