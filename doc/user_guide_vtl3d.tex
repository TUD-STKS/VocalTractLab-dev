\documentclass[]{article}
\usepackage{bm}
\usepackage{amsmath}
\usepackage{amsfonts} 
\usepackage{graphicx}
\usepackage[utf8]{inputenc}
\usepackage[T1]{fontenc}
\usepackage{mathtools}
%\usepackage{breqn}
\usepackage[left=2cm, right=2cm, top=2cm, bottom=2cm]{geometry}
\usepackage{soul}
\usepackage{color}
\usepackage{hyperref}
\usepackage{bold-extra}
\usepackage{tcolorbox}

\newcommand*{\transp}[2][-3mu]{\ensuremath{\mskip1mu\prescript{\smash{\mathrm t\mkern#1}}{}{\mathstrut#2}}}
\newcommand{\mathcolorbox}[2]{\colorbox{#1}{$\displaystyle #2$}}
\newcommand\tab[1][1cm]{\hspace*{#1}}

%opening
\title{User guide for VocalTractLab3D}
\author{R{\'e}mi Blandin}

\begin{document}
	\maketitle
	
	\section{Introduction}
	
	VocalTractLab3D is an articulatory synthesiser closely related to VocalTractLab 2.3. It has, to some very little differences, the same functionalities as 
	VocalTractLab 2.3, but it integrates a module that performs 3D acoustic 
	simulations, hence its name VocalTractLab3D.
	
	\subsection{What are 3D acoustic simulations?}
	
	Vocal tract acoustic simulations consists in descibing how acoustic waves travel
	inside the vocal tract volume, are reflected at the discontinuities, such as 
	changes of cross-section or the mouth opening and create resonances. 
	This allows one to compute transfer functions, as an exemple between the acoustic volume flow created at the vocal folds and the acoustic pressure radiated in front of the lips.
	
	Up to about 4-5 kHz, the acoustic waves consist in variations of the acoustic 
	pressure which are almost uniform over the transverse cross-section of the vocal tract. Therefore, they are often refered to as \emph{plane waves}.
	Their properties depend only on the cross-sectional area and it is not necessary to take into account the cross-sectional contour shape and the curvature of the vocal tract to describe them. 
	That is why it is sufficient to describe the vocal tract as a 1D area function 
	when plane wave propagation is assumed. 
	This is type of simulation is one-dimensional, it cannot account for transverse 
	variation of the acoustic field or of the precise 3D shape of the vocal tract.
\end{document}